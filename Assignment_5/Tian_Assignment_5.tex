\documentclass[]{article}
\usepackage{lmodern}
\usepackage{amssymb,amsmath}
\usepackage{ifxetex,ifluatex}
\usepackage{fixltx2e} % provides \textsubscript
\ifnum 0\ifxetex 1\fi\ifluatex 1\fi=0 % if pdftex
  \usepackage[T1]{fontenc}
  \usepackage[utf8]{inputenc}
\else % if luatex or xelatex
  \ifxetex
    \usepackage{mathspec}
  \else
    \usepackage{fontspec}
  \fi
  \defaultfontfeatures{Ligatures=TeX,Scale=MatchLowercase}
\fi
% use upquote if available, for straight quotes in verbatim environments
\IfFileExists{upquote.sty}{\usepackage{upquote}}{}
% use microtype if available
\IfFileExists{microtype.sty}{%
\usepackage{microtype}
\UseMicrotypeSet[protrusion]{basicmath} % disable protrusion for tt fonts
}{}
\usepackage[margin=1in]{geometry}
\usepackage{hyperref}
\hypersetup{unicode=true,
            pdfborder={0 0 0},
            breaklinks=true}
\urlstyle{same}  % don't use monospace font for urls
\usepackage{color}
\usepackage{fancyvrb}
\newcommand{\VerbBar}{|}
\newcommand{\VERB}{\Verb[commandchars=\\\{\}]}
\DefineVerbatimEnvironment{Highlighting}{Verbatim}{commandchars=\\\{\}}
% Add ',fontsize=\small' for more characters per line
\usepackage{framed}
\definecolor{shadecolor}{RGB}{248,248,248}
\newenvironment{Shaded}{\begin{snugshade}}{\end{snugshade}}
\newcommand{\AlertTok}[1]{\textcolor[rgb]{0.94,0.16,0.16}{#1}}
\newcommand{\AnnotationTok}[1]{\textcolor[rgb]{0.56,0.35,0.01}{\textbf{\textit{#1}}}}
\newcommand{\AttributeTok}[1]{\textcolor[rgb]{0.77,0.63,0.00}{#1}}
\newcommand{\BaseNTok}[1]{\textcolor[rgb]{0.00,0.00,0.81}{#1}}
\newcommand{\BuiltInTok}[1]{#1}
\newcommand{\CharTok}[1]{\textcolor[rgb]{0.31,0.60,0.02}{#1}}
\newcommand{\CommentTok}[1]{\textcolor[rgb]{0.56,0.35,0.01}{\textit{#1}}}
\newcommand{\CommentVarTok}[1]{\textcolor[rgb]{0.56,0.35,0.01}{\textbf{\textit{#1}}}}
\newcommand{\ConstantTok}[1]{\textcolor[rgb]{0.00,0.00,0.00}{#1}}
\newcommand{\ControlFlowTok}[1]{\textcolor[rgb]{0.13,0.29,0.53}{\textbf{#1}}}
\newcommand{\DataTypeTok}[1]{\textcolor[rgb]{0.13,0.29,0.53}{#1}}
\newcommand{\DecValTok}[1]{\textcolor[rgb]{0.00,0.00,0.81}{#1}}
\newcommand{\DocumentationTok}[1]{\textcolor[rgb]{0.56,0.35,0.01}{\textbf{\textit{#1}}}}
\newcommand{\ErrorTok}[1]{\textcolor[rgb]{0.64,0.00,0.00}{\textbf{#1}}}
\newcommand{\ExtensionTok}[1]{#1}
\newcommand{\FloatTok}[1]{\textcolor[rgb]{0.00,0.00,0.81}{#1}}
\newcommand{\FunctionTok}[1]{\textcolor[rgb]{0.00,0.00,0.00}{#1}}
\newcommand{\ImportTok}[1]{#1}
\newcommand{\InformationTok}[1]{\textcolor[rgb]{0.56,0.35,0.01}{\textbf{\textit{#1}}}}
\newcommand{\KeywordTok}[1]{\textcolor[rgb]{0.13,0.29,0.53}{\textbf{#1}}}
\newcommand{\NormalTok}[1]{#1}
\newcommand{\OperatorTok}[1]{\textcolor[rgb]{0.81,0.36,0.00}{\textbf{#1}}}
\newcommand{\OtherTok}[1]{\textcolor[rgb]{0.56,0.35,0.01}{#1}}
\newcommand{\PreprocessorTok}[1]{\textcolor[rgb]{0.56,0.35,0.01}{\textit{#1}}}
\newcommand{\RegionMarkerTok}[1]{#1}
\newcommand{\SpecialCharTok}[1]{\textcolor[rgb]{0.00,0.00,0.00}{#1}}
\newcommand{\SpecialStringTok}[1]{\textcolor[rgb]{0.31,0.60,0.02}{#1}}
\newcommand{\StringTok}[1]{\textcolor[rgb]{0.31,0.60,0.02}{#1}}
\newcommand{\VariableTok}[1]{\textcolor[rgb]{0.00,0.00,0.00}{#1}}
\newcommand{\VerbatimStringTok}[1]{\textcolor[rgb]{0.31,0.60,0.02}{#1}}
\newcommand{\WarningTok}[1]{\textcolor[rgb]{0.56,0.35,0.01}{\textbf{\textit{#1}}}}
\usepackage{graphicx,grffile}
\makeatletter
\def\maxwidth{\ifdim\Gin@nat@width>\linewidth\linewidth\else\Gin@nat@width\fi}
\def\maxheight{\ifdim\Gin@nat@height>\textheight\textheight\else\Gin@nat@height\fi}
\makeatother
% Scale images if necessary, so that they will not overflow the page
% margins by default, and it is still possible to overwrite the defaults
% using explicit options in \includegraphics[width, height, ...]{}
\setkeys{Gin}{width=\maxwidth,height=\maxheight,keepaspectratio}
\IfFileExists{parskip.sty}{%
\usepackage{parskip}
}{% else
\setlength{\parindent}{0pt}
\setlength{\parskip}{6pt plus 2pt minus 1pt}
}
\setlength{\emergencystretch}{3em}  % prevent overfull lines
\providecommand{\tightlist}{%
  \setlength{\itemsep}{0pt}\setlength{\parskip}{0pt}}
\setcounter{secnumdepth}{0}
% Redefines (sub)paragraphs to behave more like sections
\ifx\paragraph\undefined\else
\let\oldparagraph\paragraph
\renewcommand{\paragraph}[1]{\oldparagraph{#1}\mbox{}}
\fi
\ifx\subparagraph\undefined\else
\let\oldsubparagraph\subparagraph
\renewcommand{\subparagraph}[1]{\oldsubparagraph{#1}\mbox{}}
\fi

%%% Use protect on footnotes to avoid problems with footnotes in titles
\let\rmarkdownfootnote\footnote%
\def\footnote{\protect\rmarkdownfootnote}

%%% Change title format to be more compact
\usepackage{titling}

% Create subtitle command for use in maketitle
\providecommand{\subtitle}[1]{
  \posttitle{
    \begin{center}\large#1\end{center}
    }
}

\setlength{\droptitle}{-2em}

  \title{}
    \pretitle{\vspace{\droptitle}}
  \posttitle{}
    \author{}
    \preauthor{}\postauthor{}
    \date{}
    \predate{}\postdate{}
  

\begin{document}

\begin{Shaded}
\begin{Highlighting}[]
\KeywordTok{library}\NormalTok{(tidyverse)  }\CommentTok{#Data management}
\end{Highlighting}
\end{Shaded}

\begin{verbatim}
## Registered S3 methods overwritten by 'ggplot2':
##   method         from 
##   [.quosures     rlang
##   c.quosures     rlang
##   print.quosures rlang
\end{verbatim}

\begin{verbatim}
## -- Attaching packages ----------------------------------------------------------- tidyverse 1.2.1 --
\end{verbatim}

\begin{verbatim}
## v ggplot2 3.1.1     v purrr   0.3.2
## v tibble  2.1.1     v dplyr   0.8.1
## v tidyr   0.8.3     v stringr 1.4.0
## v readr   1.3.1     v forcats 0.4.0
\end{verbatim}

\begin{verbatim}
## -- Conflicts -------------------------------------------------------------- tidyverse_conflicts() --
## x dplyr::filter() masks stats::filter()
## x dplyr::lag()    masks stats::lag()
\end{verbatim}

\begin{Shaded}
\begin{Highlighting}[]
\KeywordTok{library}\NormalTok{(QuantPsyc)  }\CommentTok{#Regression standardized betas}
\end{Highlighting}
\end{Shaded}

\begin{verbatim}
## Loading required package: boot
\end{verbatim}

\begin{verbatim}
## Loading required package: MASS
\end{verbatim}

\begin{verbatim}
## 
## Attaching package: 'MASS'
\end{verbatim}

\begin{verbatim}
## The following object is masked from 'package:dplyr':
## 
##     select
\end{verbatim}

\begin{verbatim}
## 
## Attaching package: 'QuantPsyc'
\end{verbatim}

\begin{verbatim}
## The following object is masked from 'package:base':
## 
##     norm
\end{verbatim}

\begin{Shaded}
\begin{Highlighting}[]
\KeywordTok{library}\NormalTok{(ggpubr)     }\CommentTok{#Add regression information to graphs}
\end{Highlighting}
\end{Shaded}

\begin{verbatim}
## Loading required package: magrittr
\end{verbatim}

\begin{verbatim}
## 
## Attaching package: 'magrittr'
\end{verbatim}

\begin{verbatim}
## The following object is masked from 'package:purrr':
## 
##     set_names
\end{verbatim}

\begin{verbatim}
## The following object is masked from 'package:tidyr':
## 
##     extract
\end{verbatim}

\begin{Shaded}
\begin{Highlighting}[]
\CommentTok{# Packages for Assumption Testing}
\KeywordTok{library}\NormalTok{(Hmisc)      }\CommentTok{#rcorr function}
\end{Highlighting}
\end{Shaded}

\begin{verbatim}
## Loading required package: lattice
\end{verbatim}

\begin{verbatim}
## 
## Attaching package: 'lattice'
\end{verbatim}

\begin{verbatim}
## The following object is masked from 'package:boot':
## 
##     melanoma
\end{verbatim}

\begin{verbatim}
## Loading required package: survival
\end{verbatim}

\begin{verbatim}
## 
## Attaching package: 'survival'
\end{verbatim}

\begin{verbatim}
## The following object is masked from 'package:boot':
## 
##     aml
\end{verbatim}

\begin{verbatim}
## Loading required package: Formula
\end{verbatim}

\begin{verbatim}
## 
## Attaching package: 'Hmisc'
\end{verbatim}

\begin{verbatim}
## The following objects are masked from 'package:dplyr':
## 
##     src, summarize
\end{verbatim}

\begin{verbatim}
## The following objects are masked from 'package:base':
## 
##     format.pval, units
\end{verbatim}

\begin{Shaded}
\begin{Highlighting}[]
\KeywordTok{library}\NormalTok{(lawstat)    }\CommentTok{#runs.test for regression}
\KeywordTok{library}\NormalTok{(lmtest)     }\CommentTok{#Durbin-Watson test}
\end{Highlighting}
\end{Shaded}

\begin{verbatim}
## Loading required package: zoo
\end{verbatim}

\begin{verbatim}
## 
## Attaching package: 'zoo'
\end{verbatim}

\begin{verbatim}
## The following objects are masked from 'package:base':
## 
##     as.Date, as.Date.numeric
\end{verbatim}

\begin{Shaded}
\begin{Highlighting}[]
\KeywordTok{library}\NormalTok{(car)        }\CommentTok{#Variance Inflation Factor}
\end{Highlighting}
\end{Shaded}

\begin{verbatim}
## Loading required package: carData
\end{verbatim}

\begin{verbatim}
## 
## Attaching package: 'car'
\end{verbatim}

\begin{verbatim}
## The following object is masked from 'package:lawstat':
## 
##     levene.test
\end{verbatim}

\begin{verbatim}
## The following object is masked from 'package:boot':
## 
##     logit
\end{verbatim}

\begin{verbatim}
## The following object is masked from 'package:dplyr':
## 
##     recode
\end{verbatim}

\begin{verbatim}
## The following object is masked from 'package:purrr':
## 
##     some
\end{verbatim}

\begin{Shaded}
\begin{Highlighting}[]
\KeywordTok{library}\NormalTok{(MASS)       }\CommentTok{#stepAIC function for stepwise regression (but no details)}
\KeywordTok{library}\NormalTok{(olsrr)      }\CommentTok{#Stepwise regression functions}
\end{Highlighting}
\end{Shaded}

\begin{verbatim}
## 
## Attaching package: 'olsrr'
\end{verbatim}

\begin{verbatim}
## The following object is masked from 'package:MASS':
## 
##     cement
\end{verbatim}

\begin{verbatim}
## The following object is masked from 'package:datasets':
## 
##     rivers
\end{verbatim}

\begin{Shaded}
\begin{Highlighting}[]
\CommentTok{## Assumption Checking for Regression}
\NormalTok{df <-}\StringTok{ }\KeywordTok{na.omit}\NormalTok{(airquality)}

\NormalTok{mod2 <-}\StringTok{ }\KeywordTok{lm}\NormalTok{(Ozone }\OperatorTok{~}\StringTok{ }\NormalTok{Solar.R }\OperatorTok{+}\StringTok{ }\NormalTok{Wind }\OperatorTok{+}\StringTok{ }\NormalTok{Temp, }\DataTypeTok{data =}\NormalTok{ df)}
\KeywordTok{summary}\NormalTok{(mod2)}
\end{Highlighting}
\end{Shaded}

\begin{verbatim}
## 
## Call:
## lm(formula = Ozone ~ Solar.R + Wind + Temp, data = df)
## 
## Residuals:
##     Min      1Q  Median      3Q     Max 
## -40.485 -14.219  -3.551  10.097  95.619 
## 
## Coefficients:
##              Estimate Std. Error t value Pr(>|t|)    
## (Intercept) -64.34208   23.05472  -2.791  0.00623 ** 
## Solar.R       0.05982    0.02319   2.580  0.01124 *  
## Wind         -3.33359    0.65441  -5.094 1.52e-06 ***
## Temp          1.65209    0.25353   6.516 2.42e-09 ***
## ---
## Signif. codes:  0 '***' 0.001 '**' 0.01 '*' 0.05 '.' 0.1 ' ' 1
## 
## Residual standard error: 21.18 on 107 degrees of freedom
## Multiple R-squared:  0.6059, Adjusted R-squared:  0.5948 
## F-statistic: 54.83 on 3 and 107 DF,  p-value: < 2.2e-16
\end{verbatim}

\begin{Shaded}
\begin{Highlighting}[]
\CommentTok{#Assumption 1: Are the dependent variables continuous? Yes}
\KeywordTok{str}\NormalTok{(df)}
\end{Highlighting}
\end{Shaded}

\begin{verbatim}
## 'data.frame':    111 obs. of  6 variables:
##  $ Ozone  : int  41 36 12 18 23 19 8 16 11 14 ...
##  $ Solar.R: int  190 118 149 313 299 99 19 256 290 274 ...
##  $ Wind   : num  7.4 8 12.6 11.5 8.6 13.8 20.1 9.7 9.2 10.9 ...
##  $ Temp   : int  67 72 74 62 65 59 61 69 66 68 ...
##  $ Month  : int  5 5 5 5 5 5 5 5 5 5 ...
##  $ Day    : int  1 2 3 4 7 8 9 12 13 14 ...
##  - attr(*, "na.action")= 'omit' Named int  5 6 10 11 25 26 27 32 33 34 ...
##   ..- attr(*, "names")= chr  "5" "6" "10" "11" ...
\end{verbatim}

\begin{Shaded}
\begin{Highlighting}[]
\CommentTok{#Assumption 2: Are dependent y values independent from each other? Yes}
\CommentTok{#Logic}
\end{Highlighting}
\end{Shaded}

\begin{Shaded}
\begin{Highlighting}[]
\CommentTok{#Assumption 3: Non-zero variance of predictors Yes}
\KeywordTok{options}\NormalTok{(}\DataTypeTok{scipen =} \DecValTok{9999}\NormalTok{)}
\KeywordTok{apply}\NormalTok{(df, }\DecValTok{2}\NormalTok{, var)}
\end{Highlighting}
\end{Shaded}

\begin{verbatim}
##       Ozone     Solar.R        Wind        Temp       Month         Day 
## 1107.290090 8308.742179   12.657324   90.820311    2.171007   75.815233
\end{verbatim}

\begin{Shaded}
\begin{Highlighting}[]
\CommentTok{#Assumption 4: The regression model is linear in predictors? Yes}
\NormalTok{Hmisc}\OperatorTok{::}\KeywordTok{rcorr}\NormalTok{(}\KeywordTok{as.matrix}\NormalTok{(df), }\DataTypeTok{type =} \StringTok{"pearson"}\NormalTok{)}
\end{Highlighting}
\end{Shaded}

\begin{verbatim}
##         Ozone Solar.R  Wind  Temp Month   Day
## Ozone    1.00    0.35 -0.61  0.70  0.14 -0.01
## Solar.R  0.35    1.00 -0.13  0.29 -0.07 -0.06
## Wind    -0.61   -0.13  1.00 -0.50 -0.19  0.05
## Temp     0.70    0.29 -0.50  1.00  0.40 -0.10
## Month    0.14   -0.07 -0.19  0.40  1.00 -0.01
## Day     -0.01   -0.06  0.05 -0.10 -0.01  1.00
## 
## n= 111 
## 
## 
## P
##         Ozone  Solar.R Wind   Temp   Month  Day   
## Ozone          0.0002  0.0000 0.0000 0.1346 0.9569
## Solar.R 0.0002         0.1835 0.0017 0.4398 0.5471
## Wind    0.0000 0.1835         0.0000 0.0408 0.6032
## Temp    0.0000 0.0017  0.0000        0.0000 0.3134
## Month   0.1346 0.4398  0.0408 0.0000        0.9253
## Day     0.9569 0.5471  0.6032 0.3134 0.9253
\end{verbatim}

\begin{Shaded}
\begin{Highlighting}[]
\CommentTok{#Assumption 5: No perfect multicollinearity? Yes}
\NormalTok{car}\OperatorTok{::}\KeywordTok{vif}\NormalTok{(mod2) }\CommentTok{#Only works for multiple variables}
\end{Highlighting}
\end{Shaded}

\begin{verbatim}
##  Solar.R     Wind     Temp 
## 1.095253 1.329070 1.431367
\end{verbatim}

\begin{Shaded}
\begin{Highlighting}[]
\CommentTok{#Assumption 6: Highly influential points (Cook's distance)? As shown in Figure Residuals vs Leverage,all points stays within the extreme bounds. }
\CommentTok{#Assumption 7: Homoscedasticity? As shown in Figure Residuals vs Fitted, no obvious trendation exist.}
\CommentTok{#Assumption 8: Normality of residuals? As shown in Figure Normal Q-Q, it can be recognised as narmality.}
\KeywordTok{par}\NormalTok{(}\DataTypeTok{mfrow =} \KeywordTok{c}\NormalTok{(}\DecValTok{2}\NormalTok{,}\DecValTok{2}\NormalTok{))  }\CommentTok{#Set plotting window to a 2x2 orientation}
\KeywordTok{plot}\NormalTok{(mod2)           }\CommentTok{#Plot all regression plots}
\end{Highlighting}
\end{Shaded}

\includegraphics{Tian_Assignment_5_files/figure-latex/unnamed-chunk-7-1.pdf}

\begin{Shaded}
\begin{Highlighting}[]
\KeywordTok{par}\NormalTok{(}\DataTypeTok{mfrow =} \KeywordTok{c}\NormalTok{(}\DecValTok{1}\NormalTok{,}\DecValTok{1}\NormalTok{))  }\CommentTok{#Set plotting window back to single}
\CommentTok{# Plot1: Homoescedasticity. Are the residuals equal at every level?}
\CommentTok{# Plot2: Normality of residuals}
\CommentTok{# Plot3: Standardized homoscedasticity}
\CommentTok{# Plot4: Cook's distance}
\end{Highlighting}
\end{Shaded}

\begin{Shaded}
\begin{Highlighting}[]
\CommentTok{#Assumption 9: Independence of residuals? Yes.}
\NormalTok{stats}\OperatorTok{::}\KeywordTok{acf}\NormalTok{(mod2}\OperatorTok{$}\NormalTok{residuals)         }\CommentTok{#Plot for lag function: Is there a pattern in the lag? Is it predictable?}
\end{Highlighting}
\end{Shaded}

\includegraphics{Tian_Assignment_5_files/figure-latex/unnamed-chunk-8-1.pdf}

\begin{Shaded}
\begin{Highlighting}[]
\NormalTok{lawstat}\OperatorTok{::}\KeywordTok{runs.test}\NormalTok{(mod2}\OperatorTok{$}\NormalTok{residuals) }\CommentTok{#Runs test: Do the residuals differ from a straight line?}
\end{Highlighting}
\end{Shaded}

\begin{verbatim}
## 
##  Runs Test - Two sided
## 
## data:  mod2$residuals
## Standardized Runs Statistic = -0.66665, p-value = 0.505
\end{verbatim}

\begin{Shaded}
\begin{Highlighting}[]
\NormalTok{lmtest}\OperatorTok{::}\KeywordTok{dwtest}\NormalTok{(mod2)               }\CommentTok{#Durbin-Watson Test: Is there first order autocorrelation? 1.5-2.5 = normal.}
\end{Highlighting}
\end{Shaded}

\begin{verbatim}
## 
##  Durbin-Watson test
## 
## data:  mod2
## DW = 1.9355, p-value = 0.3347
## alternative hypothesis: true autocorrelation is greater than 0
\end{verbatim}

\begin{Shaded}
\begin{Highlighting}[]
\CommentTok{#Assumption 10: The mean of residuals is zero? Yes}
\KeywordTok{mean}\NormalTok{(mod2}\OperatorTok{$}\NormalTok{residuals)}
\end{Highlighting}
\end{Shaded}

\begin{verbatim}
## [1] 0.00000000000000007417701
\end{verbatim}

\begin{Shaded}
\begin{Highlighting}[]
\CommentTok{#Assumption 11: X variables and residuals are uncorrelated? Yes}

\NormalTok{df.res <-}\StringTok{ }\KeywordTok{data.frame}\NormalTok{(df, mod2}\OperatorTok{$}\NormalTok{residuals) }\OperatorTok\StringTok{ }
\StringTok{  }\NormalTok{dplyr}\OperatorTok{::}\KeywordTok{select}\NormalTok{(}\OperatorTok{-}\KeywordTok{c}\NormalTok{(Month, Day))}
\NormalTok{Hmisc}\OperatorTok{::}\KeywordTok{rcorr}\NormalTok{(}\KeywordTok{as.matrix}\NormalTok{(df.res), }\DataTypeTok{type =} \StringTok{"pearson"}\NormalTok{)}
\end{Highlighting}
\end{Shaded}

\begin{verbatim}
##                Ozone Solar.R  Wind  Temp mod2.residuals
## Ozone           1.00    0.35 -0.61  0.70           0.63
## Solar.R         0.35    1.00 -0.13  0.29           0.00
## Wind           -0.61   -0.13  1.00 -0.50           0.00
## Temp            0.70    0.29 -0.50  1.00           0.00
## mod2.residuals  0.63    0.00  0.00  0.00           1.00
## 
## n= 111 
## 
## 
## P
##                Ozone  Solar.R Wind   Temp   mod2.residuals
## Ozone                 0.0002  0.0000 0.0000 0.0000        
## Solar.R        0.0002         0.1835 0.0017 1.0000        
## Wind           0.0000 0.1835         0.0000 1.0000        
## Temp           0.0000 0.0017  0.0000        1.0000        
## mod2.residuals 0.0000 1.0000  1.0000 1.0000
\end{verbatim}

\begin{Shaded}
\begin{Highlighting}[]
\CommentTok{#Assumption 12: The number of observations must be greater than the number of Xs? Yes.}
\CommentTok{#Logic}
\end{Highlighting}
\end{Shaded}


\end{document}
